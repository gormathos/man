\begin{problem}{Pandas: Timestamp}{input.csv}{}{2 секунди}{256 мегабайт}

Timestamp часто визначається як кількість секунд, що пройшли з певного початкового моменту в часі, який відомий як \t{Epoch}. 
У багатьох системах Unix і POSIX Epoch починається з 1 січня 1970 року 00:00:00 UTC.

Таким чином, Timestamp $-$ це представлення часу у формі числа, яке відображає, 
скільки секунд пройшло з 1 січня 1970 року до певного моменту в часі. 
Цей спосіб представлення часу став стандартом у багатьох системах, що працюють з Unix, 
і використовується в програмуванні, базах даних, системному адмініструванні та інших галузях.

Напишіть програму для обробки фрагменту таблиці бази даних, який збережено у файл в форматі CSV.
Таблиця містить інформацію про відправлені в тестуючу систему DOTS розв'язки задач: 
\t{id} рішення, \t{id} користувача, \t{id} задачі, результат тестування, час відправки в \t{timestamp}.
Ваша програма має знайти кількість розв'язків, які пройшли всі тести, та вивести час відправки цих рішень в форматі
\t{dd.mm.yyyy hh:mm:ss}.

\InputFile
Файл \t{<<input.csv>>} містить такі поля, які розділяються символом \t{<<;>>}:
\begin{itemize}
  \setlength\itemsep{0em}
  \item \t{<<solution\_id>>} $-$ \t{id} рішення;
  \item \t{<<user\_id>>} $-$ \t{id} користувача;
  \item \t{<<task\_id>>} $-$ \t{id} задачі;
  \item \t{<<posted\_time>>} $-$ час відправки в \t{timestamp};
  \item \t{<<score>>} $-$ результат тестування (дійсне число в діапазоні від $0.00$ до $100.00$). 
\end{itemize}

\OutputFile
В першому рядку виведіть кількість розв'язків, які пройшли всі тести, тобто результат тестування дорівнює $100.00$.
Далі виведіть час відправки цих рішень в форматі \t{dd.mm.yyyy hh:mm:ss}.

\Example
\begin{example}
\exmpfile{01.in}{01.out}%
\end{example}

\end{problem}

