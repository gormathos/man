\begin{problem}{}{}{}{0.2 seconds}{256 megabytes}
% Numpy: poly


Let $P$ $-$ be a polynomial of degree $n$, i.e.
$$
  P(x) = c_n \cdot x^n + c_{n-1} \cdot x^{n-1} + c_{n-2} \cdot x^{n-2} + \dots + c_{1} \cdot x^{1} + c_{0} 
$$


You know that this polynomial has exactly $n$ real roots.
Write a program using the \texttt{Numpy} library that
\begin{enumerate}
 \item Find the coefficients $c_n, c_{n-1}, c_{n-2}, c_{1}, c_{0}$.
 \item Calculate the values of the given polynomial at the points $t_1, t_2, ..., t_k$.
 \item Calculate the values of the derivative of a given polynomial at the points $t_1, t_2, ..., t_k$.
\end{enumerate}




\InputFile
The first line of input specifies the number $n$ $-$ the degree of the polynomial.
The second line contains $n$ real numbers $x_1, x_2, ..., x_k$ $-$ roots of the polynomial.

The third line contains $k$ $-$ the number of points at which to compute the values of the polynomial.
The fourth line contains $k$ real numbers $t_1, t_2, ..., t_k$ $-$ points at which the values of the polynomial should be calculated.


\OutputFile
In the first line print $n$ real numbers $c_n,\; c_{n-1}, \; c_{n-2}, \; ..., \; c_{1}, \; c_{0}$ $-$ coefficients of the polynomial.
In the second line print $k$ of real numbers: $P(t_1), \; P(t_2),\; ...,\; P(t_k)$.
In the third line print $k$ real numbers: $P^\prime(t_1),\; P^\prime(t_2), \; ..., \; P^\prime(t_k)$.


The absolute error of computation should not exceed $10^{-5}$.


\Constraints
$2 \le n \le 10$.

$-10.0 \le x_i \le 10.0$.

$1 \le k \le $10.

$10.0 \le t_i \le $10.0.

\Examples
\begin{example}
\exmpfile{01.in}{01.out}%
\exmpfile{02.in}{02.out}%
\end{example}


\end{problem}

